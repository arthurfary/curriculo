
\documentclass{resume}
\usepackage{ragged2e}

\makeatletter
\renewcommand{\raggedright}{}
\makeatother

\begin{document}
\fontfamily{ppl}\selectfont

\begin{center}
    {\large Arthur Eduardo Fary} \\
    {\small
    
    \href{mailto:arthurfary22@gmail.com}{E-Mail} 
    \textbf{·}
    \href{https://github.com/arthurfary/}{GitHub}
    \textbf{·}
    \href{https://www.linkedin.com/in/arthur-fary/}{LinkedIn}
    }
\end{center}

\justifying
\small
\parindent 12pt
Sou Bacharel em Sistemas de Informação pela Universidade do Estado de Santa Catarina (UDESC) e atuo como Desenvolvedor de Software. Possuo experiência em desenvolvimento e manutenção de sistemas, automação de processos, análise e tratamento de dados, além de suporte técnico.

Tenho conhecimento avançado em Inglês e trabalho com o idioma diariamente. Possuo um portfólio diversificado disponível no GitHub, demonstrando projetos em diversas linguagens e tecnologias. Trabalhei também como monitor acadêmico na disciplina de Linguagem de Programação, reforçando minha capacidade de comunicação, liderança técnica e apoio a estudantes.  
\\
\\
\csection{HABILIDADES}{\small
    \begin{itemize}
        \item Python, Typescript, Javascript, React.js, Next.js, Node, PostgreSQL, HTML, CSS, Java, C++ , Lua, Rust, Oracle Cloud Infrastructure, Git
        \item Inglês (Avançado)      
    \end{itemize}
}
\csection{EXPERIÊNCIA PROFISSIONAL}{\small
    \begin{itemize}
        \item\textbf{Desenvolvedor de Software}
        \newline{VJ Systems Informática Ltda}
        \newline2025 – Atual
        \begin{itemize}
            \item Desenvolvimento de scripts em Quadient Script para processamento e interpretação de dados, gerando \textit{notices} e documentos financeiros para \textit{Credit Unions} dos Estados Unidos.
            \item Criação dos documentos na plataforma Quadient, integrando dados e lógica aos templates conforme as especificações da empresa parceira norte-americana.
            \item Desenvolvimento de scripts em Python para normalização e padronização de arquivos antes do processamento.            
        \end{itemize}
            
        \item\textbf{Estagiário de Tecnologia da Informação}
        \newline{Reveev Camas e Colchões}
        \newline2024 – 2025
        \begin{itemize}
            \item Registro e atualização de cadastros no sistema interno.
            \item Automatização de tarefas com a criação de scripts e macros.
            \item Utilização de Python e Pandas para filtrar dados do sistema interno.
            \item Realização de suporte técnico geral.
        \end{itemize}
        
        \item\textbf{Bolsista monitor na disciplina Linguagem de Programação}
        \newline{Universidade do Estado de Santa Catarina - UDESC}
        \newline2023 – 2025
        \begin{itemize}
            \item Monitoramento e auxílio a estudantes do primeiro ano na matéria Linguagem de Programação.
            \item Criação e condução de aulas extracurriculares na linguagem C++.
            \item Assegurar a compreensão dos alunos na matéria.
            \newpage
        \end{itemize}
    \end{itemize}
}
\csection{FORMAÇÃO}{\small
    \begin{itemize}
        \item \textbf{Bacharelado em Sistemas de Informação}
        \newline{Universidade do Estado de Santa Catarina (UDESC)}
        \newline2022 - 2025
    \end{itemize}
}
\csection{CERTIFICADOS}{\small
    \begin{itemize}
        \item 2024 - Inglês Nível B2 - DLLE
        \item 2024 - Festival Latino-Americano de Instalação de Software Livre: Desenvolvimento de API REST com NodeJS e Express e React no WordPress
        \item 2021 - Programador Web - PRONATEC
        \item 2021 - Excel Avançado - Univille
        \item 2020 - Inglês Avançado - KNN Idiomas     
    \end{itemize}
}
\csection{PROJETOS}{\small
    \begin{itemize}
        \item\textbf{PASSMAN}
        \textit{\newline\href{https://github.com/arthurfary/passman}{https://github.com/arthurfary/passman}}
        \newline Como Trabalho de Conclusão de Curso, idealizei e implementei uma solução local, centrada ao usuário e agnóstica de ferramenta para o gerenciamento seguro de senhas.
        \begin{itemize}
            \item Definição de um formato para o armazenamento local e seguro de senhas.
            \item Desenvolvimento de um protótipo que realiza a criptografia no formato definido, tanto em linha de comando \textit{CLI} quanto em interface gráfica \textit{GUI}. 
            \item Feito com Rust, utilizando os algoritmos Argon2id e ChaCha20-Poly1305. 
        \end{itemize}
    
        \item\textbf{Site Associação Protetora dos Animais (APA)}
        \textit{\newline\href{https://www.apasbs.org}{https://www.apasbs.org}}
        \begin{itemize}
            \item Como projeto de faculdade, participei como programador primário na criação do site da APA de São Bento do Sul. 
            \item O site é responsivo e dinâmico: permite aos membros da APA adicionar e remover animais para adoção, pontos de coleta e notícias por meio de um sistema backend (Área Restrita). 
            \item Feito com NextJS (JSX), JS e CSS, integrado ao banco de dados com chamadas de API em SQL. 
        \end{itemize}
        
        \item\textbf{Otimização de faturamento das unidades de saúde}
        \begin{itemize}
            \item Participação no desenvolvimento de um projeto para a Prefeitura de Sorocaba/SP.
            \item Otimiza o faturamento das unidades de saúde através da organização de planilhas de usuários com problemas de cadastro. 
            \item Foi utilizado Python e a biblioteca Pandas para melhorar a eficiência operacional.
        \end{itemize}
    \end{itemize}
}

\end{document}

